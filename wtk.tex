\tikzset{
  invisible/.style={opacity=0},
  transparent/.style={opacity=0.2},
  visible on/.style={alt={#1{}{invisible}}},
  transparent on/.style={alt={#1{transparent}{}}},
  alt/.code args={<#1>#2#3}{%
    \alt<#1>{\pgfkeysalso{#2}}{\pgfkeysalso{#3}} % \pgfkeysalso doesn't change the path
  },
}

\begin{frame}{From Distance to Kernels}
    \begin{figure}[t]
  \centering
    \begin{tikzpicture}
      \begin{axis}[%
        axis lines = none,
        height     =  8cm,
        width      =  11.0cm,
      ]
        \addplot[Set1-A, thick, no marks] table[x=x, y=a, col sep=comma] {./data/overview_TS_uneq_length_1.csv};
        \addplot[
          Set1-B, thick, no marks,
          y filter/.expression = {
            y - 10
          },
          visible on=<1>,
          transparent on=<2->
        %   \only<2->{opacity=0.1,}
        ] table[x=x, y=a, col sep=comma] {./data/overview_TS_2.csv};
        
        \addplot[
          Set1-B, thick, no marks,
          x filter/.expression = {
            x + 130
          },
          y filter/.expression = {
            y - 10
          },
          visible on=<2>,
          transparent on=<3->
        ] table[x=x, y=a, col sep=comma] {./data/overview_TS_3.csv};
        \addplot[
          Set1-B, thick, no marks,
          x filter/.expression = {
            x + 260
          },
          y filter/.expression = {
            y - 10
          },
          visible on=<3>,
          transparent on=<4->
        ] table[x=x, y=a, col sep=comma] {./data/overview_TS_3.csv};
        \addplot[
          Set1-B, thick, no marks,
          x filter/.expression = {
            x + 390
          },
          y filter/.expression = {
            y - 10
          },
          visible on=<4>,
        ] table[x=x, y=a, col sep=comma] {./data/overview_TS_2.csv};
        % TODO create new timeseries..
        
        \draw<1>[draw opacity=0.8, line width=1pt] (50.00,90.00) -- (50.00,40.);
        \draw<2>[draw opacity=0.8, line width=1pt] (50.00,90.00) -- (180.00,40.);
        \draw<3>[draw opacity=0.8, line width=1pt] (50.00,90.00) -- (310.00,40.);
        \draw<4>[draw opacity=0.8, line width=1pt] (50.00,90.00) -- (440.00,40.);
        % \draw[draw opacity=0.8, line width=0.6pt] (5.00,0.08) -- (25.00,-9.75);
        % \draw[draw opacity=0.8, line width=0.6pt] (10.00,1.38) -- (30.00,-7.66);
        % \draw[draw opacity=0.8, line width=0.6pt] (15.00,3.95) -- (35.00,-6.24);
        % \draw[draw opacity=0.8, line width=0.6pt] (20.00,1.87) -- (60.00,-9.92);
        % \draw[draw opacity=0.8, line width=0.6pt] (25.00,0.15) -- (0.00,-9.98);
        % \draw[draw opacity=0.8, line width=0.15882352941176556pt] (30.00,-0.04) -- (70.00,-10.03);
        % \draw[draw opacity=0.8, line width=0.6pt] (35.00,0.01) -- (90.00,-10.01);
        % \draw[draw opacity=0.8, line width=0.15882352941176434pt] (40.00,-0.01) -- (65.00,-9.98);
        % \draw[draw opacity=0.8, line width=0.6pt] (45.00,-0.00) -- (10.00,-10.01);
        % \draw[draw opacity=0.8, line width=0.30588235294117666pt] (50.00,-0.01) -- (75.00,-9.98);
        % \draw[draw opacity=0.8, line width=0.45294117647058646pt] (55.00,0.00) -- (85.00,-10.01);
        % \draw[draw opacity=0.8, line width=0.6pt] (60.00,-0.00) -- (105.00,-9.99);
        % \draw[draw opacity=0.8, line width=0.6pt] (65.00,-0.01) -- (5.00,-10.03);
        % \draw[draw opacity=0.8, line width=0.4529411764705877pt] (70.00,0.01) -- (80.00,-9.96);
        % \draw[draw opacity=0.8, line width=0.15882352941176434pt] (75.00,0.02) -- (15.00,-9.99);
        % \draw[draw opacity=0.8, line width=0.30588235294117666pt] (80.00,0.38) -- (50.00,-9.60);
        % \draw[draw opacity=0.8, line width=0.30588235294117666pt] (85.00,1.56) -- (55.00,-8.53);
      \end{axis}
    \end{tikzpicture}
  \end{figure}
\end{frame}

\begin{frame}{From Distance to Kernels}
  \begin{definition}[Wasserstein time series kernel]
  \normalsize
  Let $\oneseries_i$ and $\oneseries_j$ be two time series, and $\shapelet_{i1}, \dots, \shapelet_{iU}$
  as well as $\shapelet_{j1}, \dots, \shapelet_{jV}$ be their respective
  subsequences. Moreover, let $D$ be a $U \times V$ matrix that contains
  the pairwise distances of all subsequences, such that
  %
  $D_{uv} := \dist\left(\shapelet_{iu}, \shapelet_{jv}\right)$,
  %
  where $\dist(\cdot, \cdot)$ denotes the usual Euclidean distance.
  The optimisation problem
  %
  \begin{equation}\
   \wasserstein_1\left(\oneseries_i, \oneseries_j\right) := \min_{P \in \Gamma\left(\oneseries_i, \oneseries_j\right)} \left\langle D, P \right\rangle_{\mathrm{F}},
    \label{eq:Our distance}
  \end{equation}
  %
  yields the optimal transport cost to transform $\oneseries_i$
  into $\oneseries_j$ by means of their subsequences. Then, given
  $\lambda\in\real_{> 0}$, we can define
  %
  \begin{equation}
    \Method\left(\oneseries_i, \oneseries_j\right) := \exp\left(-\lambda \wasserstein_1\left(\oneseries_i, \oneseries_j\right)\right),
    \label{eq:Our kernel}
  \end{equation}
  %
  which we refer to as our \emph{Wasserstein-based subsequence kernel};
  \end{definition}
\end{frame}

%%%%%%%%%%%%%%%%%%%%%%%%%%%%%%%%%%%%%%%%%%%%%%%%%%%%%%%%%%%%%%%%%%%%%%%%
% Algorithms
%%%%%%%%%%%%%%%%%%%%%%%%%%%%%%%%%%%%%%%%%%%%%%%%%%%%%%%%%%%%%%%%%%%%%%%%

\definecolor{mydarkblue}{rgb}{0,0.08,0.45}

\renewcommand{\algorithmicrequire}{\textbf{Input:}}
\renewcommand{\algorithmicensure}{\textbf{Output:}}
\renewcommand{\algorithmiccomment}[1]{\qquad \textcolor{ETHf}{//} \textcolor{ETHf}{#1}}

\begin{frame}{\longmethod}
    \begin{algorithm}[H]
      \footnotesize
      \caption{\longmethod}
      \begin{algorithmic}[1]
        \REQUIRE{Time series for training and testing
        $\allseries_{\text{train}}$, $\allseries_{\text{test}}$; subsequence
        length $w$; kernel weight factor $\lambda$}
        \ENSURE{$\mathcal{K}^{\,\text{train}}, \mathcal{K}^{\,\text{test}}$}
          \STATE $\shapelets^{\text{train}} \gets \textsc{Subsequences}({\allseries_{\text{train}}, w})$  \COMMENT{Extract subsequences}
          \STATE $\shapelets^{\text{test}} \gets \textsc{Subsequences}({\allseries_{\text{test}}, w})$ \COMMENT{Extract subsequences}
          \FOR{$\oneseries_i \in \allseries_{\text{train}}$}
            \FOR{$\oneseries_j \in \allseries_{\text{train}}$}
              \STATE $\mathcal{D}_{ij}^{\text{train}} \gets \wasserstein_1\left(\shapelets_i^{\text{train}}, \shapelets_j^{\text{train}}\right)$ \COMMENT{Wasserstein distance calculation (train)} %\Comment{Compute Wasserstein distance between TS using subsequences}
            \ENDFOR
          \FOR{$\oneseries_k \in \allseries_{\text{test}}$}
            \STATE $\mathcal{D}_{ik}^{\text{test}} \gets \wasserstein_1\left(\shapelets_i^{\text{train}}, \shapelets_k^{\text{test}}\right)$ \COMMENT{Wasserstein distance calculation (test)} %\Comment{Compute Wasserstein distance between TS using subsequences}
            \ENDFOR
          \ENDFOR
          \STATE $\mathcal{K}^{\,\text{train}} \gets \exp\left(-\lambda \mathcal{D}^{\text{train}} \right)$ \COMMENT{Kernel matrix calculation}
          \STATE $\mathcal{K}^{\,\text{test}} \gets \exp\left(-\lambda \mathcal{D}^{\text{test}} \right)$ \COMMENT{Kernel matrix calculation}
          \RETURN $\mathcal{K}^{\,\text{train}}, \mathcal{K}^{\,\text{test}}$
      \end{algorithmic}
    \end{algorithm}
\end{frame}